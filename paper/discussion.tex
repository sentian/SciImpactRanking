 %!TEX root = main.tex
\section*{Discussion}
Comparing the scientific impacts of papers or scholars can assist the process of making academic decisions. The question is how to carry out the comparison accurately. A benchmark specifies a certain field, a specific publication year or a single document type. Rank percentiles based on the number of citations are reasonable metrics to compare two entities in different benchmarks\supercite{bornmann2013use}. In this paper, we discuss a general framework to construct rank percentile indicators that is flexible in terms of the choice of evaluation metric. For paper impact, we introduce rp.c$_{j\tau}$ that measure the performance of paper $j$ at age $\tau$ based on the number of citations. Meanwhile, for scholar impact, we consider rp.c$_{i\tau}$, rp.h$_{i\tau}$ and rp.rp5$_{i\tau}$ that correspond to the evaluation metric being the number of citations, h-index and an aggregation of paper performance measured using rp.c$_{j\tau}$. We do not claim a single best metric that shall be applied, although we show some advantages of using rp.rp5$_{i\tau}$ over the others. Once constructed, the rank percentile indicator itself is highly interpretable. It tells the rank of a paper or a scholar relative to others in the benchmark with the same age. Hence we can compare say two scholars in the same area but starting their careers at different years. 

We further study the predictive power of the rank percentile indicators. We show that both the paper impact rp.c$_{j\tau}$ and the scholar impact rp.rp5$_{i\tau}$ have high predictive powers. Meanwhile, rp.c$_{j\tau}$ is stable over age, meaning that the ranking of a paper is likely to stay the same along different stages. Predicting cumulative impact rp.c and rp.rp5 is of interest in making academic decision such as hiring new faculties. Assigning research funding often requires foreseeing the future impact of a scholar's future works. We see that rp.rp5 still has reasonably high predictive power in predicting the future impact, although it's not as much high as predicting the cumulative impact. Furthermore, we formulate the prediction tasks into supervised learning problems. We show that an extensive list of features and complex machine learning models bring negligible improvement in predictions. Both the cumulative impact and future impact can be predicted well using simple linear regression models. 
